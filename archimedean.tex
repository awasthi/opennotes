\documentclass{article}
\usepackage{fullpage}
\usepackage{amsmath,amsfonts,amssymb,amsthm,epsfig,epstopdf,url,array}
\theoremstyle{plain}
\newtheorem{thm}{Theorem}[section]
\newtheorem{lem}[thm]{Lemma}
\newtheorem{prop}[thm]{Proposition}
\newtheorem*{cor}{Corollary}

\theoremstyle{definition}
\newtheorem*{defn}{DEFINITION}
\newtheorem{conj}{Conjecture}[section]
\newtheorem{exmp}{Example}[section]


\begin{document}
	\title{Archimedean Property}
	\maketitle
	\begin{defn}
		An ordered field $F$ has the Archimedean Property if, given any positive $x$ and
		$y$ in $F$ there is an integer $n>0$ so that $nx > y$.
	\end{defn}

	\begin{thm}
	The set of real numbers (an ordered field with the Least Upper Bound property)
	has the Archimedean Property.
	\end{thm}

\begin{lem}
	The set $N$ of positive integers $N =\{0, 1, 2, . . .\}$ is not bounded from above.
	
	Assume $N$ is bounded from above. Since $N\subset R$ and $R$ 	has the least upper bound property, then $N$ has a least upper bound $a\in R$. Thus $n\le a$ for all $n \in N$ and is the smallest such real number.
	
	Consequently $a-1$ is not an upper bound for $N$ (if it were, since $a-1<a$, then $a$ would
	not be the least upper bound). Therefore there is some integer $k$ with $a-1 < k$. But then
	$a < k + 1$. This contradicts that $a$ is an upper bound for $N$.
\end{lem}
	Proof: Since $x > 0$, the statement that there is an integer $n$ so that $nx > y$ is equivalent to finding an $n$ with $n > y/x$ for some $n$, But if there is no such $n$ then
	$n < y/x$ for all integers $n$. That is, $y/x$ would be an upper bound for the integers. This
	contradicts the Lemma.
	\qed

	Think example that will have an ordered field that does not have the Archimedean property.
\section*{Summary}
If $a$ and $b$ are positive integers, then there exists a positive number $n$, such that 
\[na \ge b\]
Example:  Let $a=7$ and $b=100$, then we can choose $n=100$ so that $100 (7) > 100$. 
\end{document}